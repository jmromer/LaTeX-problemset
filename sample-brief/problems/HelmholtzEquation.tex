\problem[The Helmholtz Equation (5pts)] % Optional problem description

Use the concepts of wave number and angular frequency in the expression of a one-dimensional classical wave that is traveling to the right and show that this wave function obeys the classical wave equation.

\solution{

\providecommand{\wave}{y} % Denote wave amplitude as y or psi

To show that the functional expression $\wave(x,t)$ of a one-dimensional, rightward-traveling classical wave obeys the classical wave equation
%
\begin{formula}[Helmholtz Equation]
  \PD[2]{\wave}{x} = \frac{1}{v_p^2} \PD[2]{\wave}{t},
\end{formula}
%
it suffices to show that $\wave(x,t)$ is a solution to the one-dimensional Helmholtz equation.

Expressed in trigonometric form, the wave function is
$\wave = A \cos(\omega t - kx)$, with angular frequency
$\omega = 2\pi f$ and wave number $k = \Frac{2\pi}{\lambda}$.

Since the wavelength
\begin{equation}
  \d \lambda = \frac{v_p}{f},
\end{equation}

$\d k = \frac{2\pi f}{v_p} = \frac{\omega}{v_p}$. The wave function is then $\d \wave = A \cos\L(\omega t - \frac{\omega}{v_p}x\R)$.

\begin{mathtable}[caption=true,title={Table One},label={one}]{ccc}
  \a & \b & \c\\
  \midrule
  \d & \ep & \t\\
\end{mathtable}

Substituting into the Helmholtz equation, we have Figure~\ref{fig1}.
\begin{figure}[ht!]
  \vspace{0pt}
  \centering
  \includegraphics[width=1in]{problems/figure.jpg}
  \caption{A rather boring figure.}
  \label{fig1}
\end{figure}
}