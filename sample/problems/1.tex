\problem[5.3.8 (5pts)]

Show that $\lambda \geq 0$ for the eigenvalue problem

\[ \diff[2]{\h}{x} + \left( \l - x^2 \right)\h = 0
\mtxt{with}
\diff{\h}{x}(0)=0 \mc \diff{\h}{x}(1)=0. \]

Is $\lambda = 0$ an eigenvalue?


\solution{

The Rayleigh quotient is

\begin{formula}[Rayleigh Quotient]
		\lambda = \frac{\eval{-p(x) \phi(x)\phi'(x)}{a}{b} + \Int{p(x) [\phi'(x)]^2 - q(x) [\phi(x)]^2}{a}{b}}{\Int{[\phi(x)]^2\sigma(x)}{a}{b}}.
\end{formula}

With $p=1$, $q=-x^2$, and $\sigma=1$, the boundary conditions imply
$\eval{-\h\h'}{0}{1}=0$,
so our eigenvalues must satisfy

\begin{equation*}
	\lambda
  = \frac{ \eval{-\phi \phi'}{0}{1}
    + \Int{[\phi']^2 + x^2 \phi^2}{0}{1} }{\Int{\phi^2}{0}{1}}
	= \frac{ \Int{[\phi']^2}{0}{1}
    + \Int{[x\phi]^2}{0}{1} }{\Int{\phi^2}{0}{1}}
	\geq 0.
\end{equation*}

Furthermore, if $\lambda = 0$ then $\Int{[\phi']^2}{0}{1} = - \Int{[x\phi]^2}{0}{1}$. That is possible only if $\phi'(x)\equiv0$, which in turn implies that $\phi(x)\equiv c$ for some constant $c$.

From the boundary conditions, it must be the case that $c=0$. But this would indicate a trivial solution, and thus $\lambda \neq 0$. So there is no zero eigenvalue.
}
